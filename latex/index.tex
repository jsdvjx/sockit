\hypertarget{index_intro_sec}{}\section{Introduction}\label{index_intro_sec}
\hyperlink{classSockIt}{SockIt} is a Chrome plugin that allows Javascript to perform asynchronous, low-\/level networking functions. Our API allows web developers to create servers and clients which perform asynchronous network I/O by using Javascript events to perform callbacks. The API provides access to both TCP and UDP protocols, as well as basic concurrency control for tweaking performance.\hypertarget{index_install_sec}{}\section{Installation}\label{index_install_sec}
Our plugin is currently packaged for 64-\/bit Chrome and Firefox on both Windows and Linux. To install our plugin, visit our \href{http://extensions/chrome/sockit/tutorial/downloads.html}{\tt downloads} page.\hypertarget{index_features_sec}{}\section{Features}\label{index_features_sec}
Our plugin allows creation of TCP and UDP clients and servers, exposes a basic API for controlling each, and allows developers to perform asynchronous I/O with this clients and servers using Javascript event callbacks. Likewise, our API allows some concurrency control to improve performance, provides hooks to handle additional events such as errors, connection, disconnection, and asynchronously logs errors on the client machine for debugging.\hypertarget{index_tutorial_sec}{}\section{Tutorial}\label{index_tutorial_sec}
To see a quick example of our plugin, see our \href{http://extensions/chrome/sockit/tutorial/tutorial.html}{\tt Hello World tutorial} or our \href{http://extensions/chrome/sockit/tutorial/full_tutorial.html}{\tt full tutorial}. 

To learn about everything that our plugin can do, see our \href{http://extensions/chrome/sockit/tutorial/api_documentation.html}{\tt API documentation}.\hypertarget{index_install_sec}{}\section{Installation}\label{index_install_sec}
To get familiar with our code, browse this full source documentation. 

The build process is automated on both Windows and Linux, but has several dependencies. For full instructions on installation can be found \href{../index.html}{\tt here}.  